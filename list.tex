%%%%%%%%%%%%%%%%%%%%%%%%%%%%%
%Since : Aug/21/2007
%Update: <Sep/09/2008>
%%%%%%%%%%%%%%%%%%%%%%%%%%%%%

\chapter{ニーモニックと機械語のリスト(アルファベット順)}
\footnotesize

\begin{center}
\begin{tabular}{|c|c|c|}
\hline
\multicolumn{3}{|c|}{ADD(加算)}\\
\hline
機械語& ニーモニック& 機能 \\ \hline
87& ADD A, A& AにAを加える\\ \hline
80& ADD A, B& AにBを加える\\ \hline
81& ADD A, C& AにCを加える\\ \hline
82& ADD A, D& AにDを加える\\ \hline
83& ADD A, E& AにEを加える\\ \hline
84& ADD A, H& AにHを加える\\ \hline
85& ADD A, L& AにLを加える\\ \hline
C6 n& ADD A, n& Aにnを加える\\ \hline
\hline
\multicolumn{3}{|c|}{AND(論理積)}\\
\hline
A7& AND A& AとAのビットごとの論理積をとる\\ \hline
A0& AND B& AとBのビットごとの論理積をとる\\ \hline
A1& AND C& AとCのビットごとの論理積をとる\\ \hline
A2& AND D& AとDのビットごとの論理積をとる\\ \hline
A3& AND E& AとEのビットごとの論理積をとる\\ \hline
A4& AND H& AとHのビットごとの論理積をとる\\ \hline
A5& AND L& AとLのビットごとの論理積をとる\\ \hline
E6 n& AND n& Aとnのビットごとの論理積をとる\\ \hline
\hline
\multicolumn{3}{|c|}{CALL(呼び出し)}\\
\hline
CD n m& CALL mnH& mnH番地をコールする \\ \hline
DC n m& CALL C, mnH& 桁上げ桁下げがあるときmnH番地をコールする\\ \hline
C4 n m& CALL NC, mnH& 桁上げ桁下げが無いときmnH番地をコールする\\ \hline
FC n m& CALL M, mnH& 負の数のときmnH番地をコールする\\ \hline
F4 n m& CALL P, mnH&  正の数のときmnH番地をコールする\\ \hline
EC n m& CALL PE, mnH& 偶数のときmnH番地をコールする\\ \hline
E4 n m& CALL PO, mnH& 奇数のときmnH番地をコールする\\ \hline
CC n m& CALL Z, mnH& ゼロのときmnH番地をコールする\\ \hline
D4 n m& CALL NZ, mnH& ゼロでないときのときmnH番地をコールする\\ \hline
\hline
\multicolumn{3}{|c|}{CP(比較)}\\
\hline
BF& CP A& オペラントとAを比較する\\ \hline
B8& CP B& \\ \hline
B9& CP C& \\ \hline
BA& CP D& \\ \hline
BB& CP E& \\ \hline
BC& CP H& \\ \hline
BD& CP L& \\ \hline
FE n& CP n& \\ \hline
\end{tabular}
\end{center}

\newpage



\begin{center}
\begin{tabular}{|c|c|c|}
\hline
\multicolumn{3}{|c|}{DEC(デクリメント)}\\
\hline
機械語& ニーモニック& 機能 \\ \hline
3D& DEC A& 指定したレジスタから1引く\\ \hline
05& DEC B& \\ \hline
0D& DEC C& \\ \hline
15& DEC D& \\ \hline
1D& DEC E& \\ \hline
25& DEC H& \\ \hline
2D& DEC L& \\ \hline
\hline
\multicolumn{3}{|c|}{INPUT(入力)}\\
\hline
DB n& IN A, (n)& nで指定したI/OポートにAを入力\\ \hline
\hline
\multicolumn{3}{|c|}{INC(インクリメント)}\\
\hline
3C& INC A& 指定したレジスタに1足す\\ \hline
04& INC B&\\ \hline
0C& INC C&\\ \hline
14& INC D&\\ \hline
1C& INC E&\\ \hline
24& INC H&\\ \hline
2C& INC L&\\ \hline
\hline
\multicolumn{3}{|c|}{JP(ジャンプ)}\\
\hline
C3 n m& JP mnH& mnH番地にジャンプする\\ \hline
DA n m& JP C, mnH& 桁上げ桁下げフラグが立っているときmnH番地にジャンプする\\ \hline
D2 n m& JP NC, mnH& 桁上げ桁下げフラグが立っていないときmnH番地にジャンプする\\ \hline
FA n m& JP M, mnH& 負の数フラグが立っているときmnH番地にジャンプする\\ \hline
F2 n m& JP P, mnH& 正の数フラグが立っているときmnH番地にジャンプする\\ \hline
EA n m& JP PE, mnH& 偶数フラグが立っているときmnH番地にジャンプする\\ \hline
E2 n m& JP PO, mnH& 奇数フラグが立っているときmnH番地にジャンプする\\ \hline
CA n m& JP Z, mnH& ゼロフラグ立っているときmnH番地にジャンプする\\ \hline
C2 n m& JP NZ, mnH& ゼロフラグがたっていないときmnH番地にジャンプする\\ \hline
\hline
\multicolumn{3}{|c|}{LD(転送)}\\
\hline
3A n m& LD A, (mnH)&  メモリmnH番地の値をAに転送する\\ \hline
7F& LD A, A & AをAに転送する  \\ \hline
78& LD A, B & BをAに転送する  \\ \hline
79& LD A, C & CをAに転送する \\ \hline
7A& LD A, D & DをAに転送する \\ \hline
7B& LD A, E & EをAに転送する \\ \hline
7C& LD A, H & HをAに転送する \\ \hline
7D& LD A, L & LをAに転送する \\ \hline
3E n&  LD A, n  & 値nをAに転送する\\ \hline
47& LD B, A & AをBに転送する  \\ \hline
40& LD B, B & BをBに転送する  \\ \hline
41& LD B, C & CをBに転送する  \\ \hline
42& LD B, D & DをBに転送する  \\ \hline
43& LD B, E & EをBに転送する  \\ \hline
44& LD B, H & HをBに転送する  \\ \hline
45& LD B, L & LをBに転送する  \\ \hline
06 n& LD B, n & 値nをBに転送する  \\ \hline
32 nm& LD mn, A & AをmnHに転送する  \\ \hline
\end{tabular}
\end{center}


\newpage



\begin{center}
\begin{tabular}{|c|c|c|}
\hline
\multicolumn{3}{|c|}{LD(転送)}\\
\hline
機械語& ニーモニック& 機能 \\ \hline
4F& LD C, A & AをCに転送する  \\ \hline
48& LD C, B & BをCに転送する  \\ \hline
49& LD C, C & CをCに転送する  \\ \hline
4A& LD C, D & DをCに転送する  \\ \hline
4B& LD C, E & EをCに転送する  \\ \hline
4C& LD C, H & HをCに転送する  \\ \hline
4D& LD C, L & LをCに転送する  \\ \hline
0E n& LD C, n & 値nをCに転送する  \\ \hline
57& LD D, A & AをDに転送する  \\ \hline
50& LD D, B & BをDに転送する  \\ \hline
51& LD D, C & CをDに転送する  \\ \hline
52& LD D, D & DをDに転送する  \\ \hline
53& LD D, E & EをDに転送する  \\ \hline
54& LD D, H & HをDに転送する  \\ \hline
55& LD D, L & LをDに転送する  \\ \hline
16 n& LD D, n & 値nをDに転送する  \\ \hline
5F& LD E, A & AをEに転送する  \\ \hline
58& LD E, B & BをEに転送する  \\ \hline
59& LD E, C & CをEに転送する  \\ \hline
5A& LD E, D & DをEに転送する  \\ \hline
5B& LD E, E & EをEに転送する  \\ \hline
5C& LD E, H & HをEに転送する  \\ \hline
5D& LD E, L & LをEに転送する  \\ \hline
1E n& LD E, n & 値nをEに転送する  \\ \hline
67& LD H, A & AをHに転送する  \\ \hline
60& LD H, B & BをHに転送する  \\ \hline
61& LD H, C & CをHに転送する  \\ \hline
62& LD H, D & DをHに転送する  \\ \hline
63& LD H, E & EをHに転送する  \\ \hline
64& LD H, H & HをHに転送する  \\ \hline
65& LD H, L & LをHに転送する  \\ \hline
26 n& LD H, n & 値nをHに転送する  \\ \hline
6F& LD L, A & AをLに転送する  \\ \hline
68& LD L, B & BをLに転送する  \\ \hline
69& LD L, C & CをLに転送する  \\ \hline
6A& LD L, D & DをLに転送する  \\ \hline
6B& LD L, E & EをLに転送する  \\ \hline
6C& LD L, H & HをLに転送する  \\ \hline
6D& LD L, L & LをLに転送する  \\ \hline
2E n& LD L, n & 値nをLに転送する  \\ \hline
01 n m& LD BC, mn & 値mnHをBCに転送する  \\ \hline
11 n m& LD DE, mn & 値mnHをDEに転送する  \\ \hline
21 n m& LD HL, mn & 値mnHをHLに転送する  \\ \hline

\hline

\end{tabular}
\end{center}




\newpage



\begin{center}
\begin{tabular}{|c|c|c|}

\hline
\multicolumn{3}{|c|}{OR(論理和)}\\
\hline
機械語& ニーモニック& 機能 \\ \hline
B7& OR A& AとAのビットごとの論理和をとる\\ \hline
B0& OR B& AとBのビットごとの論理和をとる\\ \hline
B1& OR C& AとCのビットごとの論理和をとる\\ \hline
B2& OR D& AとDのビットごとの論理和をとる\\ \hline
B3& OR E& AとEのビットごとの論理和をとる\\ \hline
B4& OR H& AとHのビットごとの論理和をとる\\ \hline
B5& OR L& AとLのビットごとの論理和をとる\\ \hline
F6 n& OR  n& Aとnのビットごとの論理和をとる \\ \hline
\hline
\multicolumn{3}{|c|}{OUT(出力)}\\
\hline
D3 n& OUT (n), A & Aの内容をI/Oポートnに出力する \\ \hline
\hline
\multicolumn{3}{|c|}{RET(戻る)}\\
\hline
C9 &RET & サブルーティンから戻る \\ \hline
D8 &RET C& 桁上げ桁下げフラグが立っているときサブルーティンから戻る \\ \hline
D0 &RET NC& 桁上げ桁下げフラグが立っていないときサブルーティンから戻る \\ \hline
F8 &RET M& 負の数フラグが立っているときサブルーティンから戻る \\ \hline
F0 &RET P& 正の数フラグが立っているときサブルーティンから戻る \\ \hline
E8 &RET PE& 偶数フラグが立っているときサブルーティンから戻る \\ \hline
E0 &RET PO& 奇数フラグが立っているときサブルーティンから戻る \\ \hline
C8 &RET Z& ゼロフラグが立っているときサブルーティンから戻る \\ \hline
C0 &RET NZ& ゼロフラグが立っていないときサブルーティンから戻る \\ \hline
\hline
\multicolumn{3}{|c|}{RLCA(左ビットシフト)}\\
\hline
07&  RLCA& Aの内容を左にシフトする\\ \hline
\hline
\multicolumn{3}{|c|}{RRCA(右ビットシフト)}\\
\hline
0F&  RRCA& Aの内容を右にシフトする\\ \hline
\hline
\multicolumn{3}{|c|}{SUB(減算)}\\
\hline
97 & SUB A & AからAを引く\\ \hline
90 & SUB B & AからBを引く\\ \hline
91 & SUB C & AからCを引く\\ \hline
92 & SUB D & AからDを引く\\ \hline
93 & SUB E & AからEを引く\\ \hline
94 & SUB H & AからHを引く\\ \hline
95 & SUB L & AからLを引く\\ \hline
D6 n & SUB n & Aからnを引く\\ \hline
\hline
\multicolumn{3}{|c|}{XOR(排他的論理和)}\\
\hline
AF &XOR A& AとAのビットごとの排他的論理和をとる \\ \hline
A8 &XOR B& AとBのビットごとの排他的論理和をとる\\ \hline
A9 &XOR C& AとCのビットごとの排他的論理和をとる\\ \hline
AA &XOR D& AとDのビットごとの排他的論理和をとる\\ \hline
AB &XOR E& AとEのビットごとの排他的論理和をとる\\ \hline
AC &XOR H& AとHのビットごとの排他的論理和をとる\\ \hline
AD &XOR L& AとLのビットごとの排他的論理和をとる\\ \hline
EE n&XOR n& Aとnのビットごとの排他的論理和をとる\\ \hline
\end{tabular}
\end{center}
