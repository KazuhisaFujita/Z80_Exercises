%%%%%%%%%%%%%%%%%%%%%%%%%%%%%
%Update: <Sep/14/2007>
%%%%%%%%%%%%%%%%%%%%%%%%%%%%%
\chapter{マイクロコンピュータ応用}

\section{目的}

マイクロコンピュータ演習の集大成として、これまで習った知識を用いてステッピングモー
タ制御のシステムを構築する。

\section{装置}

マイコントレーナーMT-Zおよびステッピングモータを用いる。

\section{最終課題}
以下の機能を実装するプログラムを作りなさい。細かい仕様は各グループの判断に任せま
す。レポートではプログラムの仕様(どのような動作を行うプログラムか)、大まかな処理の流
れ、プログラムソースを報告すること。

\begin{itemize}
\item ステッピングモータインターフェースをパラレルIOボードにつなぎ、1相励磁回転させる。
\item スイッチにより、回転方向、回転速度を変えられるようにする。
\item LEDの点灯を回転に応じて変化させる。(変化のさせかたは任意。例えば回転速度をLED
  の点灯数で表す、回転に応じてLEDの点灯が移動するなど)
\end{itemize}
